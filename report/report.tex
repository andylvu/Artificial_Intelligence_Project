%% bare_jrnl_compsoc.tex
%% V1.4b
%% 2022/03/28
%% by Michael Shell
%% See:
%% http://www.michaelshell.org/
%% for current contact information.
%%
%% This is a skeleton file demonstrating the use of IEEEtran.cls
%% (requires IEEEtran.cls version 1.8b or later) with an IEEE
%% Computer Society journal paper.
%%
%% Support sites:
%% http://www.michaelshell.org/tex/ieeetran/
%% http://www.ctan.org/pkg/ieeetran
%% and
%% http://www.ieee.org/

%%*************************************************************************
%% Legal Notice:
%% This code is offered as-is without any warranty either expressed or
%% implied; without even the implied warranty of MERCHANTABILITY or
%% FITNESS FOR A PARTICULAR PURPOSE! 
%% User assumes all risk.
%% In no event shall the IEEE or any contributor to this code be liable for
%% any damages or losses, including, but not limited to, incidental,
%% consequential, or any other damages, resulting from the use or misuse
%% of any information contained here.
%%
%% All comments are the opinions of their respective authors and are not
%% necessarily endorsed by the IEEE.
%%
%% This work is distributed under the LaTeX Project Public License (LPPL)
%% ( http://www.latex-project.org/ ) version 1.3, and may be freely used,
%% distributed and modified. A copy of the LPPL, version 1.3, is included
%% in the base LaTeX documentation of all distributions of LaTeX released
%% 2003/12/01 or later.
%% Retain all contribution notices and credits.
%% ** Modified files should be clearly indicated as such, including  **
%% ** renaming them and changing author support contact information. **
%%*************************************************************************


% *** Authors should verify (and, if needed, correct) their LaTeX system  ***
% *** with the testflow diagnostic prior to trusting their LaTeX platform ***
% *** with production work. The IEEE's font choices and paper sizes can   ***
% *** trigger bugs that do not appear when using other class files.       ***                          ***
% The testflow support page is at:
% http://www.michaelshell.org/tex/testflow/

\documentclass[10pt,journal,compsoc]{IEEEtran}
%
% If IEEEtran.cls has not been installed into the LaTeX system files,
% manually specify the path to it like:
% \documentclass[10pt,journal,compsoc]{../sty/IEEEtran}

% Some very useful LaTeX packages include:
% (uncomment the ones you want to load)

% *** MISC UTILITY PACKAGES ***
%
%\usepackage{ifpdf}
% Heiko Oberdiek's ifpdf.sty is very useful if you need conditional
% compilation based on whether the output is pdf or dvi.
% usage:
% \ifpdf
%   % pdf code
% \else
%   % dvi code
% \fi
% The latest version of ifpdf.sty can be obtained from:
% http://www.ctan.org/pkg/ifpdf
% Also, note that IEEEtran.cls V1.7 and later provides a builtin
% \ifCLASSINFOpdf conditional that works the same way.
% When switching from latex to pdflatex and vice-versa, the compiler may
% have to be run twice to clear warning/error messages.

% *** CITATION PACKAGES ***
%
\ifCLASSOPTIONcompsoc
  % IEEE Computer Society needs nocompress option
  % requires cite.sty v4.0 or later (November 2003)
  \usepackage[nocompress]{cite}
  \usepackage{graphicx}
\else
  % normal IEEE
  \usepackage{cite}.
  
\fi
% cite.sty was written by Donald Arseneau
% V1.6 and later of IEEEtran pre-defines the format of the cite.sty package
% \cite{} output to follow that of the IEEE. Loading the cite package will
% result in citation numbers being automatically sorted and properly
% "compressed/ranged". e.g., [1], [9], [2], [7], [5], [6] without using
% cite.sty will become [1], [2], [5]--[7], [9] using cite.sty. cite.sty's
% \cite will automatically add leading space, if needed. Use cite.sty's
% noadjust option (cite.sty V3.8 and later) if you want to turn this off
% such as if a citation ever needs to be enclosed in parenthesis.
% cite.sty is already installed on most LaTeX systems. Be sure and use
% version 5.0 (2009-03-20) and later if using hyperref.sty.
% The latest version can be obtained at:
% http://www.ctan.org/pkg/cite
% The documentation is contained in the cite.sty file itself.
%
% Note that some packages require special options to format as the Computer
% Society requires. In particular, Computer Society  papers do not use
% compressed citation ranges as is done in typical IEEE papers
% (e.g., [1]-[4]). Instead, they list every citation separately in order
% (e.g., [1], [2], [3], [4]). To get the latter we need to load the cite
% package with the nocompress option which is supported by cite.sty v4.0
% and later. Note also the use of a CLASSOPTION conditional provided by
% IEEEtran.cls V1.7 and later.

% *** GRAPHICS RELATED PACKAGES ***
%
\ifCLASSINFOpdf
  % \usepackage[pdftex]{graphicx}
  % declare the path(s) where your graphic files are
  % \graphicspath{{../pdf/}{../jpeg/}}
  % and their extensions so you won't have to specify these with
  % every instance of \includegraphics
  % \DeclareGraphicsExtensions{.pdf,.jpeg,.png}
\else
  % or other class option (dvipsone, dvipdf, if not using dvips). graphicx
  % will default to the driver specified in the system graphics.cfg if no
  % driver is specified.
  % \usepackage[dvips]{graphicx}
  % declare the path(s) where your graphic files are
  % \graphicspath{{../eps/}}
  % and their extensions so you won't have to specify these with
  % every instance of \includegraphics
  % \DeclareGraphicsExtensions{.eps}
\fi
% graphicx was written by David Carlisle and Sebastian Rahtz. It is
% required if you want graphics, photos, etc. graphicx.sty is already
% installed on most LaTeX systems. The latest version and documentation
% can be obtained at: 
% http://www.ctan.org/pkg/graphicx
% Another good source of documentation is "Using Imported Graphics in
% LaTeX2e" by Keith Reckdahl which can be found at:
% http://www.ctan.org/pkg/epslatex
%
% latex, and pdflatex in dvi mode, support graphics in encapsulated
% postscript (.eps) format. pdflatex in pdf mode supports graphics
% in .pdf, .jpeg, .png and .mps (metapost) formats. Users should ensure
% that all non-photo figures use a vector format (.eps, .pdf, .mps) and
% not a bitmapped formats (.jpeg, .png). The IEEE frowns on bitmapped formats
% which can result in "jaggedy"/blurry rendering of lines and letters as
% well as large increases in file sizes.
%
% You can find documentation about the pdfTeX application at:
% http://www.tug.org/applications/pdftex

% *** MATH PACKAGES ***
%
%\usepackage{amsmath}
% A popular package from the American Mathematical Society that provides
% many useful and powerful commands for dealing with mathematics.
%
% Note that the amsmath package sets \interdisplaylinepenalty to 10000
% thus preventing page breaks from occurring within multiline equations. Use:
%\interdisplaylinepenalty=2500
% after loading amsmath to restore such page breaks as IEEEtran.cls normally
% does. amsmath.sty is already installed on most LaTeX systems. The latest
% version and documentation can be obtained at:
% http://www.ctan.org/pkg/amsmath

% *** SPECIALIZED LIST PACKAGES ***
%
%\usepackage{algorithmic}
% algorithmic.sty was written by Peter Williams and Rogerio Brito.
% This package provides an algorithmic environment fo describing algorithms.
% You can use the algorithmic environment in-text or within a figure
% environment to provide for a floating algorithm. Do NOT use the algorithm
% floating environment provided by algorithm.sty (by the same authors) or
% algorithm2e.sty (by Christophe Fiorio) as the IEEE does not use dedicated
% algorithm float types and packages that provide these will not provide
% correct IEEE style captions. The latest version and documentation of
% algorithmic.sty can be obtained at:
% http://www.ctan.org/pkg/algorithms
% Also of interest may be the (relatively newer and more customizable)
% algorithmicx.sty package by Szasz Janos:
% http://www.ctan.org/pkg/algorithmicx

% *** ALIGNMENT PACKAGES ***
%
%\usepackage{array}
% Frank Mittelbach's and David Carlisle's array.sty patches and improves
% the standard LaTeX2e array and tabular environments to provide better
% appearance and additional user controls. As the default LaTeX2e table
% generation code is lacking to the point of almost being broken with
% respect to the quality of the end results, all users are strongly
% advised to use an enhanced (at the very least that provided by array.sty)
% set of table tools. array.sty is already installed on most systems. The
% latest version and documentation can be obtained at:
% http://www.ctan.org/pkg/array

% IEEEtran contains the IEEEeqnarray family of commands that can be used to
% generate multiline equations as well as matrices, tables, etc., of high
% quality.

% *** SUBFIGURE PACKAGES ***
%\ifCLASSOPTIONcompsoc
%  \usepackage[caption=false,font=footnotesize,labelfont=sf,textfont=sf]{subfig}
%\else
%  \usepackage[caption=false,font=footnotesize]{subfig}
%\fi
% subfig.sty, written by Steven Douglas Cochran, is the modern replacement
% for subfigure.sty, the latter of which is no longer maintained and is
% incompatible with some LaTeX packages including fixltx2e. However,
% subfig.sty requires and automatically loads Axel Sommerfeldt's caption.sty
% which will override IEEEtran.cls' handling of captions and this will result
% in non-IEEE style figure/table captions. To prevent this problem, be sure
% and invoke subfig.sty's "caption=false" package option (available since
% subfig.sty version 1.3, 2005/06/28) as this is will preserve IEEEtran.cls
% handling of captions.
% Note that the Computer Society format requires a sans serif font rather
% than the serif font used in traditional IEEE formatting and thus the need
% to invoke different subfig.sty package options depending on whether
% compsoc mode has been enabled.
%
% The latest version and documentation of subfig.sty can be obtained at:
% http://www.ctan.org/pkg/subfig

% *** FLOAT PACKAGES ***
%
%\usepackage{fixltx2e}
% fixltx2e, the successor to the earlier fix2col.sty, was written by
% Frank Mittelbach and David Carlisle. This package corrects a few problems
% in the LaTeX2e kernel, the most notable of which is that in current
% LaTeX2e releases, the ordering of single and double column floats is not
% guaranteed to be preserved. Thus, an unpatched LaTeX2e can allow a
% single column figure to be placed prior to an earlier double column
% figure.
% Be aware that LaTeX2e kernels dated 2015 and later have fixltx2e.sty's
% corrections already built into the system in which case a warning will
% be issued if an attempt is made to load fixltx2e.sty as it is no longer
% needed.
% The latest version and documentation can be found at:
% http://www.ctan.org/pkg/fixltx2e

%\usepackage{stfloats}
% stfloats.sty was written by Sigitas Tolusis. This package gives LaTeX2e
% the ability to do double column floats at the bottom of the page as well
% as the top. (e.g., "\begin{figure*}[!b]" is not normally possible in
% LaTeX2e). It also provides a command:
%\fnbelowfloat
% to enable the placement of footnotes below bottom floats (the standard
% LaTeX2e kernel puts them above bottom floats). This is an invasive package
% which rewrites many portions of the LaTeX2e float routines. It may not work
% with other packages that modify the LaTeX2e float routines. The latest
% version and documentation can be obtained at:
% http://www.ctan.org/pkg/stfloats
% Do not use the stfloats baselinefloat ability as the IEEE does not allow
% \baselineskip to stretch. Authors submitting work to the IEEE should note
% that the IEEE rarely uses double column equations and that authors should try
% to avoid such use. Do not be tempted to use the cuted.sty or midfloat.sty
% packages (also by Sigitas Tolusis) as the IEEE does not format its papers in
% such ways.
% Do not attempt to use stfloats with fixltx2e as they are incompatible.
% Instead, use Morten Hogholm'a dblfloatfix which combines the features
% of both fixltx2e and stfloats:
%
% \usepackage{dblfloatfix}
% The latest version can be found at:
% http://www.ctan.org/pkg/dblfloatfix

%\ifCLASSOPTIONcaptionsoff
%  \usepackage[nomarkers]{endfloat}
% \let\MYoriglatexcaption\caption
% \renewcommand{\caption}[2][\relax]{\MYoriglatexcaption[#2]{#2}}
%\fi
% endfloat.sty was written by James Darrell McCauley, Jeff Goldberg and 
% Axel Sommerfeldt. This package may be useful when used in conjunction with 
% IEEEtran.cls'  captionsoff option. Some IEEE journals/societies require that
% submissions have lists of figures/tables at the end of the paper and that
% figures/tables without any captions are placed on a page by themselves at
% the end of the document. If needed, the draftcls IEEEtran class option or
% \CLASSINPUTbaselinestretch interface can be used to increase the line
% spacing as well. Be sure and use the nomarkers option of endfloat to
% prevent endfloat from "marking" where the figures would have been placed
% in the text. The two hack lines of code above are a slight modification of
% that suggested by in the endfloat docs (section 8.4.1) to ensure that
% the full captions always appear in the list of figures/tables - even if
% the user used the short optional argument of \caption[]{}.
% IEEE papers do not typically make use of \caption[]'s optional argument,
% so this should not be an issue. A similar trick can be used to disable
% captions of packages such as subfig.sty that lack options to turn off
% the subcaptions:
% For subfig.sty:
% \let\MYorigsubfloat\subfloat
% \renewcommand{\subfloat}[2][\relax]{\MYorigsubfloat[]{#2}}
% However, the above trick will not work if both optional arguments of
% the \subfloat command are used. Furthermore, there needs to be a
% description of each subfigure *somewhere* and endfloat does not add
% subfigure captions to its list of figures. Thus, the best approach is to
% avoid the use of subfigure captions (many IEEE journals avoid them anyway)
% and instead reference/explain all the subfigures within the main caption.
% The latest version of endfloat.sty and its documentation can obtained at:
% http://www.ctan.org/pkg/endfloat
%
% The IEEEtran \ifCLASSOPTIONcaptionsoff conditional can also be used
% later in the document, say, to conditionally put the References on a 
% page by themselves.

% *** PDF, URL AND HYPERLINK PACKAGES ***
%
%\usepackage{url}
% url.sty was written by Donald Arseneau. It provides better support for
% handling and breaking URLs. url.sty is already installed on most LaTeX
% systems. The latest version and documentation can be obtained at:
% http://www.ctan.org/pkg/url
% Basically, \url{my_url_here}.

% *** Do not adjust lengths that control margins, column widths, etc. ***
% *** Do not use packages that alter fonts (such as pslatex).         ***
% There should be no need to do such things with IEEEtran.cls V1.6 and later.
% (Unless specifically asked to do so by the journal or conference you plan
% to submit to, of course. )

% correct bad hyphenation here
\hyphenation{op-tical net-works semi-conduc-tor}

\begin{document}
%
% paper title
% Titles are generally capitalized except for words such as a, an, and, as,
% at, but, by, for, in, nor, of, on, or, the, to and up, which are usually
% not capitalized unless they are the first or last word of the title.
% Linebreaks \\ can be used within to get better formatting as desired.
% Do not put math or special symbols in the title.
\title{Creating an Optimal Grocery List with Budget and Recommended Food Groups\\ 7375 Artifical Intelligence}
%
%
% author names and IEEE memberships
% note positions of commas and nonbreaking spaces ( ~ ) LaTeX will not break
% a structure at a ~ so this keeps an author's name from being broken across
% two lines.
% use \thanks{} to gain access to the first footnote area
% a separate \thanks must be used for each paragraph as LaTeX2e's \thanks
% was not built to handle multiple paragraphs
%
%
%\IEEEcompsocitemizethanks is a special \thanks that produces the bulleted
% lists the Computer Society journals use for "first footnote" author
% affiliations. Use \IEEEcompsocthanksitem which works much like \item
% for each affiliation group. When not in compsoc mode,
% \IEEEcompsocitemizethanks becomes like \thanks and
% \IEEEcompsocthanksitem becomes a line break with idention. This
% facilitates dual compilation, although admittedly the differences in the
% desired content of \author between the different types of papers makes a
% one-size-fits-all approach a daunting prospect. For instance, compsoc 
% journal papers have the author affiliations above the "Manuscript
% received ..."  text while in non-compsoc journals this is reversed. Sigh.

\author{Andy~Vu,
    Sam~Hekman,
    and~Tyler~Sutherland% <-this % stops a space
    \IEEEcompsocitemizethanks{\IEEEcompsocthanksitem Department
        Computer Science, Kennesaw State University, Marietta,
        GA, 1100 South Marietta Pkwy SE 30060. \\
        git: https://github.com/vespenegas/Artificial\_Intelligence\_Project\protect\\
        % note need leading \protect in front of \\ to get a newline within \thanks as
        % \\ is fragile and will error, could use \hfil\break instead.

    }% <-this % stops an unwanted space
    \thanks{March 28, 2022}}

% note the % following the last \IEEEmembership and also \thanks - 
% these prevent an unwanted space from occurring between the last author name
% and the end of the author line. i.e., if you had this:
% 
% \author{....lastname \thanks{...} \thanks{...} }
%                     ^------------^------------^----Do not want these spaces!
%
% a space would be appended to the last name and could cause every name on that
% line to be shifted left slightly. This is one of those "LaTeX things". For
% instance, "\textbf{A} \textbf{B}" will typeset as "A B" not "AB". To get
% "AB" then you have to do: "\textbf{A}\textbf{B}"
% \thanks is no different in this regard, so shield the last } of each \thanks
% that ends a line with a % and do not let a space in before the next \thanks.
% Spaces after \IEEEmembership other than the last one are OK (and needed) as
% you are supposed to have spaces between the names. For what it is worth,
% this is a minor point as most people would not even notice if the said evil
% space somehow managed to creep in.

% The paper headers
\markboth{Journal of \LaTeX\ Class Files,~Vol.~14, No.~8, August~2015}%
{Shell \MakeLowercase{\textit{et al.}}: Bare Demo of IEEEtran.cls for Computer Society Journals}
% The only time the second header will appear is for the odd numbered pages
% after the title page when using the twoside option.
% 
% *** Note that you probably will NOT want to include the author's ***
% *** name in the headers of peer review papers.                   ***
% You can use \ifCLASSOPTIONpeerreview for conditional compilation here if
% you desire.

% The publisher's ID mark at the bottom of the page is less important with
% Computer Society journal papers as those publications place the marks
% outside of the main text columns and, therefore, unlike regular IEEE
% journals, the available text space is not reduced by their presence.
% If you want to put a publisher's ID mark on the page you can do it like
% this:
%\IEEEpubid{0000--0000/00\$00.00~\copyright~2015 IEEE}
% or like this to get the Computer Society new two part style.
%\IEEEpubid{\makebox[\columnwidth]{\hfill 0000--0000/00/\$00.00~\copyright~2015 IEEE}%
%\hspace{\columnsep}\makebox[\columnwidth]{Published by the IEEE Computer Society\hfill}}
% Remember, if you use this you must call \IEEEpubidadjcol in the second
% column for its text to clear the IEEEpubid mark (Computer Society jorunal
% papers don't need this extra clearance.)

% use for special paper notices
%\IEEEspecialpapernotice{(Invited Paper)}

% for Computer Society papers, we must declare the abstract and index terms
% PRIOR to the title within the \IEEEtitleabstractindextext IEEEtran
% command as these need to go into the title area created by \maketitle.
% As a general rule, do not put math, special symbols or citations
% in the abstract or keywords.
\IEEEtitleabstractindextext{%
    \begin{abstract}
        Developing an optimal grocery list based on budget and food group
        restraints causes the nature of the problem to identify as a
        multidimensional knapsack problem. The knapsack problem has been around
        for a longtime and still continues to cause mathematicians to struggle.
        There have been a multitude of conventional and well developed
        approaches to the knapsack problem as well as novel ideas. In this
        paper, we explore the prospects of solving the knapsack problem using a
        genetic algorithm. Although not entirely novel, we propose and explore
        the idea to introduce another dimension of food groups to add complexity
        to the knapsack. With this in mind we have devised a brute force and
        dynamic programming solutions to solve the knapsack problem alongside a
        custom generated dataset to serve as baselines to the genetic algorithm.
        The results show that conventional approaches such as brute force and
        dynamic programming work so long as the input size of the dataset is
        small but genetic algorithms work much more efficiently if there is a
        large enough dataset. Although not in the case of our implementation due
        to the small dataset our genetic algorithm was ten times slower than the
        dynamic programming but offered a much larger diversity in its output
        solutions. The idea of a multidimensional knapsack including food groups
        was found to be far too simple to even be bothered to implement and that
        datasets and grocery store design is much more simple and complex than
        perceived to be. 
    \end{abstract}

    % Note that keywords are not normally used for peerreview papers.
    \begin{IEEEkeywords}
        Computer Science, Artificial Intelligence, Knapsack, NP Complete, List
        generation, Dynamic programming, Bruteforce, Exhaustive search, Genetic
        algorithms, Heuristic algorithms \end{IEEEkeywords}}


% make the title area
\maketitle

% To allow for easy dual compilation without having to reenter the
% abstract/keywords data, the \IEEEtitleabstractindextext text will
% not be used in maketitle, but will appear (i.e., to be "transported")
% here as \IEEEdisplaynontitleabstractindextext when the compsoc 
% or transmag modes are not selected <OR> if conference mode is selected 
% - because all conference papers position the abstract like regular
% papers do.
\IEEEdisplaynontitleabstractindextext
% \IEEEdisplaynontitleabstractindextext has no effect when using
% compsoc or transmag under a non-conference mode.

% For peer review papers, you can put extra information on the cover
% page as needed:
% \ifCLASSOPTIONpeerreview
% \begin{center} \bfseries EDICS Category: 3-BBND \end{center}
% \fi
%
% For peerreview papers, this IEEEtran command inserts a page break and
% creates the second title. It will be ignored for other modes.
\IEEEpeerreviewmaketitle

\IEEEraisesectionheading{\section{Introduction}\label{sec:introduction}}
% Computer Society journal (but not conference!) papers do something unusual
% with the very first section heading (almost always called "Introduction").
% They place it ABOVE the main text! IEEEtran.cls does not automatically do
% this for you, but you can achieve this effect with the provided
% \IEEEraisesectionheading{} command. Note the need to keep any \label that
% is to refer to the section immediately after \section in the above as
% \IEEEraisesectionheading puts \section within a raised box.

% The very first letter is a 2 line initial drop letter followed
% by the rest of the first word in caps (small caps for compsoc).
% 
% form to use if the first word consists of a single letter:
% \IEEEPARstart{A}{demo} file is ....
% 
% form to use if you need the single drop letter followed by
% normal text (unknown if ever used by the IEEE):
% \IEEEPARstart{A}{}demo file is ....
% 
% Some journals put the first two words in caps:
% \IEEEPARstart{T}{his demo} file is ....
% 
% Here we have the typical use of a "T" for an initial drop letter
% and "HIS" in caps to complete the first word.
\IEEEPARstart{T}{he} knapsack problem is a classic problem that has troubled
mathematicians for well over a century \cite{mathews_partition_1896}. It can be
easily be described as given a set of items each with their own weight and
value, one is to choose items that maximizes the total value while within the
restraints of the knapsack's weight or other limitations. Depending on the
application, it is easy to see how widespread this problem can be. Whether it be
packing bags for a trip on an airline and attempting to not go over the weight
limit, to a thief stealing the most valuable goods from a store and trying to
make out with as much money as possible, this problem is difficult to solve.

In a general sense all knapsack problems are similar, there are quite some
variations to the problem, although the classic 0-1 knapsack problem is probably
the most common and popular. The 0-1 represents either an item being selected
(1) or not being selected to put into a bag (0). This means that there is only
one of each item and that there can be no more than one of that item. To an
extent, there is the bounded knapsack problem where there are a certain number
of duplicates of items and in contrast there is the unbounded knapsack problem
where there are no bounds or limitations of items. In other words there are
unlimited copies of each item available. In addition to these there is also a
fractional knapsack. The fractional knapsack allows the ability to pick a
fraction of items instead of an item whole. An example would be selecting ½ of
an item although in most scenarios this is not possible.

Since this problem is one that has troubled mathematicians for quite some time,
there are a variety of conventional approaches to the problem. As can be
imagined, increasing the variables at play allows more complexity to the
problem. The problem itself is defined as NP complete. In a classic knapsack
problem the time complexity is O(N*W) where N is the number of items available
and W denotes the capacity of the knapsack. This time complexity is obtained
using dynamic programming. Other conventional solutions such as brute force
results in creating every permutation possible and then selecting the most
optimal knapsack. This results in a time complexity of O(2\textsuperscript{n}).
As can be seen, increasing the number of items will cause more options to be
selected and thus the number of possible combinations increases. The user now
has more items that they may consider to be selected into the knapsack. On the
other hand, increasing the allowances of the knapsack results in a similar
situation. By allowing a greater number of items into the knapsack, the problem
again is an increase in the amount of items that are available for selection due
to this higher limit, thus increasing the dimensionality. Increasing these two
variables even or adding a third variable will further cause the knapsack
problem to be more complicated.


In this paper, we are approaching the knapsack problem from a different
perspective than conventionally. Firstly, instead of a typical knapsack we are
generating a grocery list with price being our constraint similar to weight. In
addition to this, instead of the value of an item, the grocery item will be
weighed, thus more weight of groceries will coincide with a higher value. To add
a layer of complexity, instead of generating a grocery list to maximize weight
and budget constraints, there is another dimension to this problem: food group
recommendations. The objective of this paper is to generate a grocery list from
a conventional grocery store that satisfies the constraints of a typical
knapsack; however it adds another dimension. Food group recommendations is to
ensure that the grocery list generated does not select only the most cost
effective and weighty item but to ensure variety in the optimal list of objects.
In other words, in this paper, we explore a multidimensional unbounded knapsack
problem within a grocery store \cite{agha_binary_2021}.

% You must have at least 2 lines in the paragraph with the drop letter
% (should never be an issue)


\hfill

\hfill March 28, 2022

% needed in second column of first page if using \IEEEpubid
%\IEEEpubidadjcol

% An example of a floating figure using the graphicx package.
% Note that \label must occur AFTER (or within) \caption.
% For figures, \caption should occur after the \includegraphics.
% Note that IEEEtran v1.7 and later has special internal code that
% is designed to preserve the operation of \label within \caption
% even when the captionsoff option is in effect. However, because
% of issues like this, it may be the safest practice to put all your
% \label just after \caption rather than within \caption{}.
%
% Reminder: the "draftcls" or "draftclsnofoot", not "draft", class
% option should be used if it is desired that the figures are to be
% displayed while in draft mode.
%
%\begin{figure}[!t]
%\centering
%\includegraphics[width=2.5in]{myfigure}
% where an .eps filename suffix will be assumed under latex, 
% and a .pdf suffix will be assumed for pdflatex; or what has been declared
% via \DeclareGraphicsExtensions.
%\caption{Simulation results for the network.}
%\label{fig_sim}
%\end{figure}

% Note that the IEEE typically puts floats only at the top, even when this
% results in a large percentage of a column being occupied by floats.
% However, the Computer Society has been known to put floats at the bottom.

% An example of a double column floating figure using two subfigures.
% (The subfig.sty package must be loaded for this to work.)
% The subfigure \label commands are set within each subfloat command,
% and the \label for the overall figure must come after \caption.
% \hfil is used as a separator to get equal spacing.
% Watch out that the combined width of all the subfigures on a 
% line do not exceed the text width or a line break will occur.
%
%\begin{figure*}[!t]
%\centering
%\subfloat[Case I]{\includegraphics[width=2.5in]{box}%
%\label{fig_first_case}}
%\hfil
%\subfloat[Case II]{\includegraphics[width=2.5in]{box}%
%\label{fig_second_case}}
%\caption{Simulation results for the network.}
%\label{fig_sim}
%\end{figure*}
%
% Note that often IEEE papers with subfigures do not employ subfigure
% captions (using the optional argument to \subfloat[]), but instead will
% reference/describe all of them (a), (b), etc., within the main caption.
% Be aware that for subfig.sty to generate the (a), (b), etc., subfigure
% labels, the optional argument to \subfloat must be present. If a
% subcaption is not desired, just leave its contents blank,
% e.g., \subfloat[].


% An example of a floating table. Note that, for IEEE style tables, the
% \caption command should come BEFORE the table and, given that table
% captions serve much like titles, are usually capitalized except for words
% such as a, an, and, as, at, but, by, for, in, nor, of, on, or, the, to
% and up, which are usually not capitalized unless they are the first or
% last word of the caption. Table text will default to \footnotesize as
% the IEEE normally uses this smaller font for tables.
% The \label must come after \caption as always.
%
%\begin{table}[!t]
%% increase table row spacing, adjust to taste
%\renewcommand{\arraystretch}{1.3}
% if using array.sty, it might be a good idea to tweak the value of
% \extrarowheight as needed to properly center the text within the cells
%\caption{An Example of a Table}
%\label{table_example}
%\centering
%% Some packages, such as MDW tools, offer better commands for making tables
%% than the plain LaTeX2e tabular which is used here.
%\begin{tabular}{|c||c|}
%\hline
%One & Two\\
%\hline
%Three & Four\\
%\hline
%\end{tabular}
%\end{table}

% Note that the IEEE does not put floats in the very first column
% - or typically anywhere on the first page for that matter. Also,
% in-text middle ("here") positioning is typically not used, but it
% is allowed and encouraged for Computer Society conferences (but
% not Computer Society journals). Most IEEE journals/conferences use
% top floats exclusively. 
% Note that, LaTeX2e, unlike IEEE journals/conferences, places
% footnotes above bottom floats. This can be corrected via the
% \fnbelowfloat command of the stfloats package.

\section{Approach}
For every knapsack problem there must be a list of selectable items with
appropriate weights, and values associated. Initially, our attempt at creating a
life like a grocery store was to find an appropriate dataset free online.
Looking through websites such as Kaggle and other databases, there were no
suitable datasets available. Every dataset lacked food categorization. In our
project we attempt to create a grocery shopping list with one of the criteria
being limitations by food groups. Thus, because of this, all of the available
datasets did not contain a categorization of the six food groups. This
expectation was quite low, but the conclusive evidence showed our predictions
were correct after going through many datasets. Secondly another issue with
datasets that grocery stores often used was that the items the stores carried
were not limited to foods. Any walk into a grocery store would show items such
as plates, utensils, cooking ware and so on. These items are not within the
scope of our project. The datasets found often had tens of thousands of these
items. The last issue with using a pre-created dataset was that often there are
many overlapping products that only differ by brand and price thus also not
within the scope of our project. An example would be orange juice. There are
many different brands of orange juice all with their own respective prices and
various sizes. 

With these issues with available datasets at hand, we formulated our own dataset
of grocery food items. For our custom dataset, we used the online shopping
utility available on Kroger.com. From here, we sort by all departments shopping
and select grocery food items that are definable within food groups. This
meaning, items that were complete meals or that combined food groups were
omitted. Examples of this would be a can of ravioli as there are plentiful
grains as well as meats within the item. We prioritized items that were
singularly within one food group such as raw meat, vegetables, or grains. Other
criteria that are selected is the price, and weight. Of course some items are
priced by unit instead of by weight thus to solve this issue, a quick google
search of the average weight was used. An example of this would be an apple that
Kroger sells for one dollar per apple. We would google the average weight of an
apple and add it to our database as price and weight. An example of our dataset
can be seen below in table \ref{table:Example of Groceries Dataset}. After
nearing 70 pages of Kroger's shopping database, and selecting certain items, as
of now our database sits at 189 entries. Limitations of brand and size were also
challenges that we faced. Entries in our database such as white bread, were
selected as the generic Kroger brand white bread with its associated price and
weight. All other forms of white bread and brands were omitted to avoid
confusion and complexity in our database as well as redundancy. As time
progresses and if there arises a necessity, the dataset will be refined with
either the addition or subtraction of items.


\begin{table}[h]
    \setlength\tabcolsep{2pt}
    \centering
    \begin{tabular}{c|c|c|c}
    \hline
    \multicolumn{4}{|c|}{Groceries}                                      \\
    \hline
    Food                    & Category & Price(dollars) & Weight(lbs) \# \\
    \hline
    chicken breast          & meat     & $11.78$        & 4.7            \\
    80\% lean ground beef   & meat     & $5.99$         & 1              \\
    banana                  & fruit    & $.23$          & .41            \\
    strawberries            & fruit    & $2.5$          & 1              \\
    cucumber                & veggie   & $.69$          & .75            \\
    large raw shrimp        & meat     & $13.98$        & 2              \\
    shredded cheddar cheese & dairy    & $2.29$         & .5             \\
    hot dog buns            & grain    & $1.49$         & .6875          \\
    white bread             & grain    & $2.75$         & 1.25           \\
    \hline
\end{tabular}
    \caption{Custom dataset collected from Kroger showing food, food group, price and weight or calculated weight. This example table is limited to ten selected entires from the dataset.}
    \label{table:Example of Groceries Dataset}
\end{table}

For baseline comparisons to the proposed model, we have implemented both a brute
force method and a dynamic programming method to tackle the knapsack. As
discussed further in the discussion sections, the current algorithms are set to
work as conventional 0-1 knapsacks disregarding food group criteria. These two
baseline algorithms serve to demonstrate the time complexity and usability in
comparison to the genetic algorithm that will be implemented in the future. Both
the brute force and dynamic programming methods take in the budget as a
constraint, using weights as the value to maximize and consider items by their
price. The brute force algorithm uses itertools combination method to create a
list of all possible combinations within the budget constraint, and then chooses
the maximum value and list that composes that combination.

Using dynamic programming for our algorithm, we intended to use a top-down
memoization approach to compute subproblems for the price limit and weight
maximization. This approach works by building a table based on the number of
items in the dataset by the limit in integers. For example, if the dataset has
200 items and the price limit is 100, then the table will have a shape of
200X100. Each index stores the datapoint of an item, or combination of items,
that yields the maximum possible weight at the current cost. The algorithm then
traverses the array to find the index with the highest weight yield at the
maximum price limit. Each datapoint in this index is added to a results list
that is returned to the user. Just as with the brute force approach, this
algorithm is subject to being a 0-1 knapsack. Any result using memoization
returns a list filled entirely with the single best price-to-weight ratio. This
is not ideal nor realistic for a practical grocery list.

Our logic for selecting the genetic algorithm was that the genetic approach of
selecting 'fitness' of data points could prove to be a better fit. Genetic
algorithms rely on randomization of selected items for a specific subset size
and compares the values of each item. These subsets are stored as 'chromosomes',
typically as a binary encoding, that have a length of the number of items. Using
a fitness function, we can find the best chromosome in the subsets to fill the
knapsack. Each encoded 1 in the chromosome corresponds to a specific item that
is added to the knapsack while the 0 items are not added. 

Lastly we have developed a user interface (UI) system that allows our program
ease of use. An example of it can be seen in figure \ref{fig:UI3}. We decided to
create a GUI system using Tkinter, an import within pythons' built-in library,
for our algorithm to help users interact with the system more easily. Using its
different widgets such as Labels to display text, input areas to allow the user
to input their desired budget amount, check boxes to enable or disable certain
food groups and lastly a button to submit the users order. Also we used
different windows to allow us to create an easy operating experience for the
user.

\begin{figure}[h]
    \centering
    \textbf{Example User Interface}
    \includegraphics[width=\columnwidth]{assets/finalVersionUI3.JPG}
    \caption{Example userinterface allow the user to select food restrictions as well as inputting budget}
    \label{fig:UI3}
\end{figure}

\section{Literature Review}
“On the Partition of Numbers” by G. B. Mathews explores the multichoice,
multi-dimensional knapsack problem back in 1896 as one of the earliest to do so.
Although it was not known as the knapsack problem in his work, it does propose a
problem of finding a set of non-negative integers that are a multivariate number
with N number of assignments. Mathews concludes that the formula solution for
some set of integers would be the same solution as used for a different set of
integers \cite{mathews_partition_1896}.

Caserta, et al. propose tackling the multiple choice multidimensional knapsack
using a lagrangian based scheme. Their technique combines lagrangian relaxation
with the corridor method. Their results show the most competitive approaches for
the nominal multiple choice multidimensional knapsack however their goal was to
investigate the trade off between solution reliability and robustness price in
hopes to bring theory and practice in operations research world
\cite{caserta_robust_2019}.

“A hybrid genetic algorithm for the multidimensional knapsack problem” by F.
Djannaty and S. Doostdar was an interesting read on the development of a genetic
algorithm to work with MD knapsack. We looked at this article to see what it
would take to create a hybrid GA to solve knapsack. Looked at the constraints
they used for mutations and cross-breeding, as well as the functions and
formulas they used to create the algorithm \cite{djannaty_hybrid_2008}. 

“A hybrid heuristic for the multiple choice multidimensional knapsack problem”
by R. Mansi, C. Alves, C. Valerio de, and S. Hanafi.  The aim of this article
was to see how we could set up constraints to separate our food groups into
different classes, so that we could run our algorithms. It was also our aim to
see how a heuristic algorithm would operate for this problem. When looking into
the testing they add one item from each class into their knapsack. When it came
to the heuristic approach they refined the upper and lower bounds of their
algorithm. Creating multiple stages in their algorithm to adjust to different
inputs. Having cutting, fixation, and reformulation phases as well as multiple
others used to calculate their solution \cite{mansi_hybrid_2013}.

“Neural Knapsack: A Neural Network Based Solver for the Knapsack Problem'' by
Nomer, Alnowibet, Elsayed, and Mohamed proposes a neural network and deep
learning based heuristic for solving the knapsack problem. This is accomplished
by keeping the relationships of items to the capacity based on previous attempts
and leveraging the ability to utilize the distribution of the data. As such, the
proposed model generalizes based on the correlation between values and weights
of the items in the set and their history of being selected. The authors find
that this method was able to outperform the greedy algorithm approach using
their synthetic testing sets \cite{nomer_neural_2020}.

\section{Results}
To serve as a baseline, we have implemented a brute force algorithm that works
on our dataset to select items that result in the heaviest shopping cart while
staying under budget. From the runtimes, it is clear that brute force is
extremely slow. We truncate our grocery dataset to collect runtimes with a
budget of 100 dollars. Initially with a list length of 10 items, the run time is
relatively quick at 0.0015 seconds. However, this exponential increase in time
complexity is seen once the grocery list is increased. 20 items results in a run
time of 1.57 seconds and then finally 28 items results in a runtime of 531
seconds. Any larger list of items were not included due to the time complexity
clearly showing exponential growth. In addition to this, attempts at 30 item
list length result in run times far too long and hence we decided to stop at 28
items. The data for these run times can be seen in table
\ref{table:bruteforce_results.tex}. In addition to this, the runtime chart and
data can be seen plotted in figure \ref{fig:bruteforce_timecomplexity.png}.

\begin{table}[h]
    \setlength\tabcolsep{2pt}
    \centering
    \begin{tabular}{c|c}
    \hline
    \multicolumn{2}{|c|}{Brute Force Run Time} \\
    \hline
    length of list & runtime (seconds)         \\
    \hline
    10             & $0.0015$                  \\
    13             & $0.011$                   \\
    15             & $0.047$                   \\
    17             & $0.185$                   \\
    20             & $1.57$                    \\
    21             & $3.31$                    \\
    25             & $61.14$                   \\
    26             & $131.35$                  \\
    27             & $268.46$                  \\
    28             & $531.08$                  \\
    \hline
\end{tabular}
    \caption{Runtime data for brute force algorithm with a truncated dataset. Dataset with more than 28 items were resulting in extremely long runtimes.}
    \label{table:bruteforce_results.tex}
\end{table}

\begin{figure}[h]
    \centering
    \textbf{Brute Force Runtime Chart}
    \includegraphics[width=\columnwidth]{assets/bruteforce_runtime.png}
    \caption{Plotted runtime chart for the brute force algorithm}
    \label{fig:bruteforce_timecomplexity.png}
\end{figure}

Testing using the memoization technique with the dynamic programming showed a
linear increase in computation time versus the brute force technique, with a
growth order of O(n*m) where n is the number of items in the dataset and m is
the price limit of the grocery list. As can be seen in figure
\ref{fig:dynamic_timecomplexity.png} the run time is much quicker compared to
the brute force. In addition to this, it was able to clear the entire data set
while the brute force was not. This can be seen more clearly in table
\ref{table:dynamic_runtimes.tex}. For comparison, the bruteforce runtime for a
list of 20 items was 1.57 seconds while the dynamic programming took 0.000997
for the same list length. This is a much more rapid solution.

\begin{table}[h]
    \setlength\tabcolsep{2pt}
    \centering
    \begin{tabular}{c|c}
    \hline
    \multicolumn{2}{|c|}{Dynamic (memoization) Run Time} \\
    \hline
    length of list & runtime (seconds)                   \\
    \hline
    20             & $0.000997$                          \\
    40             & $0.00199$                           \\
    60             & $0.00199$                           \\
    80             & $0.00299$                           \\
    100            & $0.00399$                           \\
    120            & $0.00498$                           \\
    140            & $0.00499$                           \\
    160            & $0.00598$                           \\
    180            & $0.00598$                           \\
    189            & $0.00698$                           \\
    \hline
\end{tabular}
    \caption{Dynamic programming runtimes of custom dataset}
    \label{table:dynamic_runtimes.tex}
\end{table}

\begin{figure}[h]
    \centering
    \textbf{Dynamic Programming Runtime Chart}
    \includegraphics[width=\columnwidth]{assets/dynamic_runtime.png}
    \caption{Plotted runtime chart for the dynamic programming algorithm}
    \label{fig:dynamic_timecomplexity.png}
\end{figure}

The genetic algorithm was able to clear the entirety of the custom dataset
similarly to the dynamic programming. Despite this, across the board the genetic
algorithm had a slower runtime than the dynamic programming by a factor of ten.
However, despite the slower performance, the genetic algorithm did not have an
increase in runtimes similar to the dynamic programming. As can be seen in table
\ref{table:genetic_runtimes.tex}, there are certain larger list lengths that
have a quicker runtime than smaller list lengths. The time difference between
runtimes of varying list lengths did not increase by much while the other two
baseline algorithms had a clear increase in runtimes as the dataset size
increased. These results and runtimes can be seen in figure
\ref{fig:genetic_timecomplexity.png}. There are situations where the output of
the genetic algorithm surpasses the amount set by the budget restraint and these
are due to the inner workings of the numpy library as well as rounding issues.
There are times where the budget restriction is set to a price such as 35, and
the total output list is 35.14. 

\begin{table}[h]
    \setlength\tabcolsep{2pt}
    \centering
    \begin{tabular}{c|c}
    \hline
    \multicolumn{2}{|c|}{Genetic Algorithm Run Times} \\
    \hline
    length of list & runtime (seconds)                   \\
    \hline
    20             & $0.03191$                           \\
    40             & $0.04088$                           \\
    60             & $0.04685$                           \\
    80             & $0.03191$                           \\
    100            & $0.04886$                           \\
    120            & $0.04985$                           \\
    140            & $0.05384$                           \\
    160            & $0.05684$                           \\
    180            & $0.04986$                           \\
    189            & $0.05483$                           \\
    \hline
\end{tabular}
    \caption{Genetic runtimes of custom dataset}
    \label{table:genetic_runtimes.tex}
\end{table}

\begin{figure}[h]
    \centering
    \textbf{Genetic Runtime Chart}
    \includegraphics[width=\columnwidth]{assets/genetic_runtime.png}
    \caption{Plotted runtime chart for the genetic algorithm}
    \label{fig:genetic_timecomplexity.png}
\end{figure}

\section{Discussion and Future Works}
Two baseline conventional algorithms have been developed for the project: brute
force and dynamic programming. These two methods are tried and true however
their limitations in time complexity as well as being resource dependent show
that their usage is limited. For datasets that are small as seen in the tables
in the preliminary results section, they have satisfactory performance, however,
for larger datasets they are not optimal and are not ideal. For this reason, we
implemented a heuristic and nonconventional algorithm while considering other
unconventional methods such as neural networks or other advanced designes with
upper and lower bounds\cite{nomer_neural_2020} \cite{mansi_hybrid_2013}.

Looking at the genetic algorithm it is clear that is being given a disadvantage
by the small dataset. Because the dataset is only composed of 189 items, the
dynamic programming algorithm is able to clear it much quicker than the genetic
algorithm. However, if the dataset were to be much larger such as a real world
retail grocery store dataset with over 10,000 items, the genetic algorithm would
be able to clear the dataset much more quickly. This is due to the advantage of
the genetic algorithm to run on a heuristic and only being time limited based on
its ability to process its fitness function and generations. On the other hand
the conventional dynamic programming algorithm would be forced to create a table
of the entire dataset and go through each and every item. Another benefit that
the genetic algorithm was able to provide over the other two algorithms was that
it introduced a level of diversity. Because it works by introducing random
populations of items, the pre-selected items are diversified in their weights,
prices, and food types whereas the dynamic and brute force algorithms will
output the same results and answers each time. 

Plans to increase the dataset size were also considered but not executed due to
the limitations and reasoning of creating the custom dataset in the first place.
Because of the amount of redundancy and item overlap of traditional large chain
grocery stores we were not able to increase the amount of produce and meat or
dairy products in our dataset. There are only a certain amount of raw food items
that can be considered without taking in the account of brand and sizes.
Increasing our dataset to a larger size would result in redundancy of multiple
brands and sizes of the same items.

In addition to the dataset problem, there were many other design issues that our
project implementation faced. As can be seen in our brute force and dynamic
programming we are utilizing these algorithms as 0-1 knapsacks without
consideration of food groups which was a third dimension in our original design.
The reason for this is that a conventional grocery store would behave as an
unbounded knapsack. Of course there are not infinite amounts of inventory,
however, a typical grocery store inventory is substantial enough that to a
typical shopper, it would not be feasible to empty the stock of a certain item.
With this being said, designing the knapsack as unbounded would result in less
choices that the algorithm would choose. It would simply pick an item that
resulted in the maximum weight to price ratio and then select that item until
the budget is met. An example would be selecting water melons which on average
weigh roughly 25 pounds. In the custom groceries dataset, watermelon is listed
as 22 pounds with a price of 6.99. It would be too trivial for the algorithm to
select watermelons until whatever budget is met thus resulting in the most
efficient shopping cart.

Introducing our assigned ratios of food groups would simply extend this issue.
By forcing the algorithm to select a certain ratio of grains, meats, and dairy
and so on, it would select items on a similar premise. For fruit, it would only
select watermelons, for meats it would select an item that has the most
efficient weight to price item as well. These would repeat until the budget is
met and all food groups are to be accounted for. This can be combated by
dividing the initial budget into the ratios of the food groups and then running
the genetic algorithm in parallel with a reduced budget. Once this is complete
the list would be combined. However, we also decided to opt out of this
implementation as it is not a true multi-dimensional knapsack with food groups
being a third dimension. This simply is just subdividing the budget and then
performing the genetic algorithm with a subdivided dataset multiple times over
for each food group. This does not introduce a new dimension or complexity to
the problem. Another aspect of the food group design is the difficulty to
quantify the food items. How many servings is a bag of rice or within one water
melon? These aspects make it difficult to divide the food groups into ratios
that are to be added to the algorithm. With that being said, there are some
design choices that need to be made and reconsidered for the future of this
endeavor and that the initial design of the project was met with many overly
complicated ideas that were actually too simple to solve. 

\section{Peer Session Reviews}

Each group member picked a presentation and wrote about what they found
interesting about the group's project. One of the presentations that we found
ourselves interested in was the Solving Sudoku Puzzle using Deep Learning. We
looked into this one to see what deep learning technique they were using to
accomplish their goal state. When looking at their implementation process we saw
that they implemented CNN model in order to solve their puzzle which took about
three seconds to complete. It was interesting to see how they implemented CNN in
the fact they had to re-shape after flattening their model.

Our second presentation that we decided to look into was the Classification of
Star project. One of our members did a similar project in another course that
also attempted to classify stars by stellar types, although using a different
type of classification. Stars are typically classified by two methods of
classification. Heil used the Morgan-Keenan scale spectral types, which are
seven types from O to M, based on the estimated temperature of the star derived
from emission spectra readings. As alluded to in Heil's presentation, another
classification type is the Yerkes type based on the evolutionary states of
stars, such as dwarfs and giants. Heil used decision tree and SVM models to
classify his data. Because the clustering of the data should be pretty well
defined, it would make sense to use these models. This is especially the case
for SVM as, depending on the type, is a discriminative model for finding the
decision boundary of a class. Heil's approach for his data was to use a dataset
from Kaggle that had feature engineering already done. This is advantageous as
this feature engineering requires in depth knowledge of how professional
astronomers use the raw data. For example, it already has temperature defined,
instead of being typically derived from the intersection of blue and violet
filtered channels of the electro-magnetic spectrum.

Our third presentation that we picked was a presentation about implementing
digit recognition, using neural networks as algorithm solver. When looking into
their implementation it was interesting to see that they implemented KNN and CNN
models. When looking at their results it was interesting to see that KNN and CNN
model accuracies were about the same when it came to single digits. But when it
comes to multiple numbers it seems from the data results that a CNN model is
needed for getting the correct number. It was also interesting to see what
dataset they were using to develop their project. Using two different sets of
data to look at two potential  problems and trying to solve both of them using
one algorithm.

\section{Conclusion}
Our implementation of brute force to approach the knapsack problem aligns with
the conventional train of thought. Run times were incredibly slow especially as
the size of the dataset increases resulting in an exponential time complexity
growth. In addition to this, our dynamic programming approach also matches what
can be found with literature. Dynamic programming allows a much more rapid
solution to the knapsack in comparison to the brute force algorithm and
surprisingly faster than a genetic algorithm given this small dataset size.
Unfortunately we were not able to provide a dataset that allowed the genetic
algorithm to show its true strengths in comparison to conventional algorithms.
With these three we have developed a UI system for ease of use.

We were unable to implement the second half of our project implementation due to
various philosophical design aspects. Unbound knapsacks exist and are realistic
representations of grocery stores while this knapsack design is not applicable
for any of the algorithms that we developed. In addition to this, food groups as
a third dimensional is nearly undefinable and face similar problems of the
unbound knapsack with a limited amount of selection and diversity. For these
multitude of reasons we have decided to alter our implementation and must
consider further design choices in the future. 


% if have a single appendix:
%\appendix[Proof of the Zonklar Equations]
% or
%\appendix  % for no appendix heading
% do not use \section anymore after \appendix, only \section*
% is possibly needed

% use appendices with more than one appendix
% then use \section to start each appendix
% you must declare a \section before using any
% \subsection or using \label (\appendices by itself
% starts a section numbered zero.)
%

% you can choose not to have a title for an appendix
% if you want by leaving the argument blank

% Can use something like this to put references on a page
% by themselves when using endfloat and the captionsoff option.
\ifCLASSOPTIONcaptionsoff
    \newpage
\fi

% trigger a \newpage just before the given reference
% number - used to balance the columns on the last page
% adjust value as needed - may need to be readjusted if
% the document is modified later
%\IEEEtriggeratref{8}
% The "triggered" command can be changed if desired:
%\IEEEtriggercmd{\enlargethispage{-5in}}

% references section

% can use a bibliography generated by BibTeX as a .bbl file
% BibTeX documentation can be easily obtained at:
% http://mirror.ctan.org/biblio/bibtex/contrib/doc/
% The IEEEtran BibTeX style support page is at:
% http://www.michaelshell.org/tex/ieeetran/bibtex/
%\bibliographystyle{IEEEtran}
% argument is your BibTeX string definitions and bibliography database(s)
%\bibliography{IEEEabrv,../bib/paper}
%
% <OR> manually copy in the resultant .bbl file
% set second argument of \begin to the number of references
% (used to reserve space for the reference number labels box)
\bibliographystyle{IEEEtran}
\bibliography{references}

% biography section
% 
% If you have an EPS/PDF photo (graphicx package needed) extra braces are
% needed around the contents of the optional argument to biography to prevent
% the LaTeX parser from getting confused when it sees the complicated
% \includegraphics command within an optional argument. (You could create
% your own custom macro containing the \includegraphics command to make things
% simpler here.)
%\begin{IEEEbiography}[{\includegraphics[width=1in,height=1.25in,clip,keepaspectratio]{mshell}}]{Michael Shell}
% or if you just want to reserve a space for a photo:

% insert where needed to balance the two columns on the last page with
% biographies
%\newpage

% You can push biographies down or up by placing
% a \vfill before or after them. The appropriate
% use of \vfill depends on what kind of text is
% on the last page and whether or not the columns
% are being equalized.

%\vfill

% Can be used to pull up biographies so that the bottom of the last one
% is flush with the other column.
%\enlargethispage{-5in}

% that's all folks
\end{document}